\documentclass[a4paper,oneside,article, titlepage]{memoir}
\usepackage[T1]{fontenc}
\usepackage{textcomp}
\usepackage[utf8]{inputenc}
\usepackage[danish]{babel}
\usepackage[garamond]{mathdesign}
\usepackage{url}
\usepackage{graphicx}

\DeclareTextFontCommand{\textfleur}
{\fontencoding{T1}\fontfamily{FleurCornerCaps}\selectfont}


\renewcommand{\ttdefault}{pcr} % bedre typewriter font
\renewcommand{\rmdefault}{ugm} % garamond


\usepackage{lettrine}

%\overfullrule=5pt

%\setsecnumdepth{part}

\title{Testspecifikation  \\ \small{Førsteårsprojekt}}

\author
{
  Gruppe 1:\\
  Troels Henriksen (athas@sigkill.dk)\\
  Jesper Reenberg (reenberg@kampsax.dtu.dk)\\
  Martin Dybdal (dybber@dybber.dk)\\ \\
  Vejledere: Dina og Kasper
}


%\setcounter{tocdepth}{3}
%\setcounter{secnumdepth}{2}

\pagestyle{plain}

\date{\today}

\begin{document}
\maketitle
\tableofcontents*
\newpage



% Testspecifikationen SKAL beskrive:

% * planlægning af testen, herunder jeres mål med testen og hvad I
% finder særligt kritisk for jeres projekt at få testet;

% * resultaterne af testen, herunder hvilke fejl er udbedret og hvilke
% fejl findes stadig i programmet;

% * en test af funktionalitet, herunder hvordan I har genereret
% testeksempler, indholdet af testeksemplerne (fortløbende
% nummereret), resultaterne fra kørsel af testeksempler;

% * en test af brugsvenlighed, herunder en begrundelse for valg af
% evalueringsteknik, en liste over de brugsvenlighedsproblemer som er
% identificeret (fortløbende nummereret), en beskrivelses af hvordan
% de kan løses.

% Bedømmelsen af testspecifikationen fokuserer på at I: (a) har
% besvaret alle fire punkter ovenfor, (b) har planlagt testen med
% hensyntagen til jeres projekts fokus, (c) ved jeres test dækker
% samtlige krav, dvs. har mindst et testeksempel for hvert krav, (d)
% har genereret testeksempler systematisk, (e) har beskrevet løsninger
% på væsentlige funktionalitets- og brugsvenlighedsproblemer, (f) har
% lavet jeres brugsvenlighedsevaluering baseret på realistiske
% opgaver, (g) har fundet fejl ? både i funktionstesten og ved
% evalueringen af brugsvenlighed ? eller har en overbevisende
% forklaring på hvorfor ingen fejl blev fundet.

% Upload filen i jeres gruppes folder. Testspecificationen skal
% afleveres som ét dokument og oploades i pdf-format (navn
% "Testspecifikation") senest den 6. juni 2007.
 

\chapter{Strategi}
Vores test er delt i to. Den første del --- funktionstesten --- skal
afgøre om programmet kan alle de ting vi påstår det kan og at alle
disse funktioner virker korrekt. Det optimale ville være hvis denne
afprøvning kunne automatiseres, men i de fleste af testene vil det,
grundet vores output i form af menneske-orienteret HTML, ikke kunne
lade sig gøre uden at bruge mere tid på det end vi har
tilgængelig. Hvis man eksempelvis vil tjekke om (...)  For næsten alle
funktionstestene vil der være en test af ugyldig inddata, for at se om
programmet giver en brugbar fejlmeddelelse. Vi har dog valgt at samle
dette under et afsnit.


Den anden del af testen --- brugertesten --- skal bruges til at finde
ud af om brugervejledningen er forståelig og om brugeren kan finde ud
af at bruge programmet ved at læse brugermanualen. Derudover skal
testen bruges til at finde ud af om resultatsiderne er vanskelige at
læse.

\chapter{Kravtest}

Vi vil systematisk gennemgå de krav vi har stillet til vores programs
funktionalitet og beskrive hvorledes hvert enkelt krav kan
afprøves. Vi henviser til kravspecifikationen for en uddybning af
kravene, følgende afsnit vil kun omhandle hvorledes de kan testes.

\section*{Krav 1: Analyse af et helt websted}

Det testrelevante i dette krav er programmets evne til at følge links
rundt på det angivne website, og ikke at følge links der fører ud fra
websitet. Andre tests dækker hvorvidt den resulterende analyse er
korrekt, formålet med denne test er udelukkende at undersøge om alle
sider rent faktisk bliver besøgt. Det er også relevant at sikre at
sider ikke bliver analyseret mere end én gang, og at cirkulære links
(side \textit{A} linker til side \textit{B} der igen linker tilbage
til side \textit{A}). ikke forårsager at programmet går i et uendeligt
loop. Det kan ses hvilke sider der er blevet analyseret ved at se på
den genererede side-liste, hvilket betyder at denne test er afhængig
af at HTML-output (krav 6) fungerer korrekt.

\subsection{Ækvivalens-klasser}
\begin{itemize}
\item Et website der ikke indeholder links, og kun én side. Her skal
  kun den ene side besøges.
\item Et website der ikke indeholder interne links, men links til
  andre sider. Her skal kun websitets egen side besøges.
\item Et website der indeholder interne links uden cirkularitet, og
  hvor hver side kun kan findes af én sti. Her skal alle tilgængelige
  sider analyseres.
\item Et website der indeholder interne links uden cirkularitet, og
  hvor hver side kan tilgås af flere forskellige veje. Det skal testes
  at de tilgængelige sider kun analyseres én gang hver.
\item Et website der indeholder interne links med cirkulære links. Det
  skal testes at hver side kun analyseres én gang, og at programmet
  terminerer (ikke går i uendeligt loop).
\end{itemize}

\section*{Krav 1.1: Respekter robots.txt}

Programmet skal tjekke det angivne websteds robots.txt og sørge for at
sider angivet i filen ikke bliver hentet og analyseret. Det giver kun
mening at teste dette krav såfremt det er sikret at krav 1 (og krav 6)
virker korrekt. Undersøgelsen af hvilke sider der ender med at blive
analyseret kan foregå på samme måde som i krav 1.

\subsection{Ækvivalens-klasser}
\begin{itemize}
\item Et website, der ikke indeholder en robots.txt-fil, og hvor alle
  tilgængelige sider således skal analyseres.
\item Et website hvor programmet har forbud mod at besøge nogen sider,
  og hvor ingen sider således kan analyseres (bør give en
  fejlmeddelelse).
\item Et website hvor programmet har forbud mod at besøge en side, der
  har eksklusive links til undersider på det analyserede website. Den
  forbudte side, og de sider som kun den forbudte sider linker til,
  skal således ikke analyseres (og faktisk heller ikke besøges, men
  dette kan ikke testes via black-box-testing med mindre man
  analyserer serverlogs)
\item Et website hvor programmet har forbud mod at besøge en side, der
  har ikke-eksklusive links til undersider på det analyserede
  website. Den forbudte side skal således ikke analyseres, men de
  sider den linker til skal besøges såfremt der er links ti l dem fra
  andre, ikke-forbudte sider.
\end{itemize}

I de angivne tests adskilles der ikke mellem forbud hvor alle
web-crawlere har forbud mod at besøge en side, og de, hvor vores
program specifikt har forbud. Vi opfatter denne distinktion som så
teknisk at den i højere grad hører hjemme i whitebox unit-tests, især
fordi denne distinktion ikke er nævnt i krav 1.1.

\section*{Krav 1.2: Dybde af crawling}
Hvis man angiver en maksimal dybde af analysen, så skal programmet
respektere denne og gøre det korrekt.

\subsection{Baggrund}
Krav 1.2

\subsection{Ækvivalens-klasser}
\begin{itemize}
\item Ingen angivet dybde.
\item Dybde 0 --- kun én side skal analyseres.
\item Dybde 2 --- 2 links skal følges i dybden.
\item Afprøvning på et websted hvor en enkelt side bliver linket til
  fra flere steder, sådan at siden i realiteten er på flere
  dybder. Den laveste dybde skal gælde, ikke den man først finder
  siden på.
\end{itemize}

\section*{Krav 2: Analysemetoder}

\section*{Krav 2.1: Læsbarhedsindeks, LIX}

\section*{Krav 2.2: Flesch-Kincaid Readability Test (FKRT)}

\section*{Krav 2.3: Stavekontrol}

Hvis en side er angivet til dansk og en dansk ordbog til aspell er
installeret, så skal programmet kunne tjekke for stavefejl.

\subsection{Baggrund}

\subsection{Ækvivalens-klasser}
\begin{itemize}
\item aspell er ikke installeret --- Stavekontrol skal ikke udføres og
  brugeren skal have besked.
\item De korrekte sprogpakker er ikke installeret --- Stavekontrol
  skal ikke udføres på de relevante sektioner og brugeren skal have
  besked.
\item 
\end{itemize}

\section*{Krav 2.3.1: Flere sprog}

\section*{Krav 2.4: Gentagne ord}

\section*{Krav 2.5: Beregning af sidesværhedsgrad}

\section*{Krav 3: Analyse baseret på HTML-tags}

\section*{Krav 3.1: \texttt{em} og \texttt{strong}}

\section*{Krav 3.2: \texttt{hN}}

\section*{Krav 3.3: \texttt{abbr} og \texttt{acronym}}

\section*{Krav 3.4: Citater skal ikke analyseres.}

\section*{Krav 3.5: Tekst i andre sprog angivet med
  (\texttt{lang})}

\section*{Krav 3.6: \texttt{kbd}, \texttt{var} og \texttt{code}}

\section*{Krav 3.7: \texttt{bdo}}

\section*{Krav 4: Konfiguration af analyse}

\section*{Krav 5: Kommandobaseret interface}

\section*{Krav 6: Resultater i HTML-format.}

\section*{Krav 7: Platform}

\section*{Krav 8: Håndtering af HTML/XHTML}

\section*{Krav 8.1: Indkodning}

Programmet skal virke på hjemmesider indkodet med ASCII
tegnsættet. Specielt skal det testes at de dansk tegn (æøåÆØÅ) virker
korrekt.

\chapter{Yderligere funktionstests}

\section{Output-mappe}
Hvis man angiver en output-mappe

\section{Ugyldig inddata}
Ved alle de forskellige konfigurationer man kan lave er der mulighed
for at angive noget ugyldigt, f.eks. ved at indtaste tegn hvor der
forventes et tal. Denne test skal afdække om der gives fyldestgørene
fejlmeddelelser i disse tilfælde.

\subsection{Baggrund}
Der er ikke stillet krav om dette, men (...)

\chapter{Brugertest}

Til at udføre vores brugertest skal vi bruge en eller flere personer
indenfor vores målgruppe. I kravspecifikationen er vores målgruppe
specificeret som:
\begin{quote}
En person i vores målgruppe er en hjemmesideskribent der er bekendt
med HTML. Personen har en professionel interessere i at teksten er
læsbar, således at vedkommende selv vil sætte sig ind i betydningen af
de udførte analysers resultater.
\end{quote}

I kravspecifikationen står der også at programmet skal være udformet
som et kommandolinjeværktøj og at det skal kunne køre på GNU--baserede
Linux maskiner. Dette stiller herved ét yderligere krav til
forsøgspersonerne: de skal have forudgående kendskab til
kommando\-linje\-baserede programmer.

\end{document}
