\documentclass[a4paper,oneside,article, titlepage]{memoir}
\usepackage[T1]{fontenc}
\usepackage{textcomp}
\usepackage[utf8]{inputenc}
\usepackage[danish]{babel}
\usepackage[garamond]{mathdesign}
\usepackage{url}
\usepackage{graphicx}

\DeclareTextFontCommand{\textfleur}
{\fontencoding{T1}\fontfamily{FleurCornerCaps}\selectfont}


\renewcommand{\ttdefault}{pcr} % bedre typewriter font
\renewcommand{\rmdefault}{ugm} % garamond


\usepackage{lettrine}

%\overfullrule=5pt

%\setsecnumdepth{part}

\title{Testspecifikation  \\ \small{Førsteårsprojekt}}

\author
{
  Gruppe 1:\\
  Troels Henriksen (athas@sigkill.dk)\\
  Jesper Reenberg (reenberg@kampsax.dtu.dk)\\
  Martin Dybdal (dybber@dybber.dk)\\ \\
  Vejledere: Dina og Kasper
}


%\setcounter{tocdepth}{3}
%\setcounter{secnumdepth}{2}

\pagestyle{plain}

\date{\today}

\begin{document}
\maketitle
\tableofcontents*
\newpage



% Testspecifikationen SKAL beskrive:

% * planlægning af testen, herunder jeres mål med testen og hvad I
% finder særligt kritisk for jeres projekt at få testet;

% * resultaterne af testen, herunder hvilke fejl er udbedret og hvilke
% fejl findes stadig i programmet;

% * en test af funktionalitet, herunder hvordan I har genereret
% testeksempler, indholdet af testeksemplerne (fortløbende
% nummereret), resultaterne fra kørsel af testeksempler;

% * en test af brugsvenlighed, herunder en begrundelse for valg af
% evalueringsteknik, en liste over de brugsvenlighedsproblemer som er
% identificeret (fortløbende nummereret), en beskrivelses af hvordan
% de kan løses.

% Bedømmelsen af testspecifikationen fokuserer på at I: (a) har
% besvaret alle fire punkter ovenfor, (b) har planlagt testen med
% hensyntagen til jeres projekts fokus, (c) ved jeres test dækker
% samtlige krav, dvs. har mindst et testeksempel for hvert krav, (d)
% har genereret testeksempler systematisk, (e) har beskrevet løsninger
% på væsentlige funktionalitets- og brugsvenlighedsproblemer, (f) har
% lavet jeres brugsvenlighedsevaluering baseret på realistiske
% opgaver, (g) har fundet fejl ? både i funktionstesten og ved
% evalueringen af brugsvenlighed ? eller har en overbevisende
% forklaring på hvorfor ingen fejl blev fundet.

% Upload filen i jeres gruppes folder. Testspecificationen skal
% afleveres som ét dokument og oploades i pdf-format (navn
% "Testspecifikation") senest den 6. juni 2007.
 

\chapter{Strategi}
Vores test er delt i to. Den første del --- funktionstesten --- skal
afgøre om programmet kan alle de ting vi påstår det kan og at alle
disse funktioner virker korrekt. Det optimale ville være hvis denne
afprøvning kunne automatiseres, men i de fleste af testene vil det
ikke kunne lade sig gøre. Hvis man eksempelvis vil tjekke om (...)
For næsten alle funktionstestene vil der være en test af ugyldig
inddata, for at se om programmet giver en brugbar fejlmeddelelse. Vi
har dog valgt at samle dette under et afsnit.


Den anden del af testen --- brugertesten --- skal bruges til at finde
ud af om brugervejledningen er forståelig og om brugeren kan finde ud
af at bruge programmet ved at læse brugermanualen. Derudover skal
testen bruges til at finde ud af om resultatsiderne er vanskelige at
læse.

\chapter{Kravtest}

\section*{Krav 1: Analyse af et helt websted}

\section*{Krav 1.1: Respekter robots.txt}

Programmet skal tjekke det angivne websteds robots.txt og sørge for at
websider angivet i filen ikke bliver hentet og analyseret.

\section*{Krav 1.2: Dybde af crawling}
Hvis man angiver en maksimal dybde af analysen, så skal programmet
respektere denne og gøre det korrekt.

\subsection{Baggrund}
Krav 1.2

\subsection{Ækvivalens-klasser}
\begin{itemize}
\item Ingen angivet dybde.
\item Dybde 0 --- kun én side skal analyseres.
\item Dybde 2 --- 2 links skal følges i dybden.
\item Afprøvning på et websted hvor en enkelt side bliver linket til
  fra flere steder, sådan at siden i realiteten er på flere
  dybder. Den laveste dybde skal gælde, ikke den man først finder
  siden på.
\end{itemize}

\section*{Krav 2: Analysemetoder}

\section*{Krav 2.1: Læsbarhedsindeks, LIX}

\section*{Krav 2.2: Flesch-Kincaid Readability Test (FKRT)}

\section*{Krav 2.3: Stavekontrol}

Hvis en side er angivet til dansk og en dansk ordbog til aspell er
installeret, så skal programmet kunne tjekke for stavefejl.

\subsection{Baggrund}

\subsection{Ækvivalens-klasser}
\begin{itemize}
\item aspell er ikke installeret --- Stavekontrol skal ikke udføres og
  brugeren skal have besked.
\item De korrekte sprogpakker er ikke installeret --- Stavekontrol
  skal ikke udføres på de relevante sektioner og brugeren skal have
  besked.
\item 
\end{itemize}

\section*{Krav 2.3.1: Flere sprog}

\section*{Krav 2.4: Gentagne ord}

\section*{Krav 2.5: Beregning af sidesværhedsgrad}

\section*{Krav 3: Analyse baseret på HTML-tags}

\section*{Krav 3.1: \texttt{em} og \texttt{strong}}

\section*{Krav 3.2: \texttt{hN}}

\section*{Krav 3.3: \texttt{abbr} og \texttt{acronym}}

\section*{Krav 3.4: Citater skal ikke analyseres.}

\section*{Krav 3.5: Tekst i andre sprog angivet med
  (\texttt{lang})}

\section*{Krav 3.6: \texttt{kbd}, \texttt{var} og \texttt{code}}

\section*{Krav 3.7: \texttt{bdo}}

\section*{Krav 4: Konfiguration af analyse}

\section*{Krav 5: Kommandobaseret interface}

\section*{Krav 6: Resultater i HTML-format.}

\section*{Krav 7: Platform}

\section*{Krav 8: Håndtering af HTML/XHTML}

\section*{Krav 8.1: Indkodning}

Programmet skal virke på hjemmesider indkodet med ASCII
tegnsættet. Specielt skal det testes at de dansk tegn (æøåÆØÅ) virker
korrekt.

\chapter{Yderligere funktionstests}

\section{Output-mappe}
Hvis man angiver en output-mappe

\section{Ugyldig inddata}
Ved alle de forskellige konfigurationer man kan lave er der mulighed
for at angive noget ugyldigt, f.eks. ved at indtaste tegn hvor der
forventes et tal. Denne test skal afdække om der gives fyldestgørene
fejlmeddelelser i disse tilfælde.

\subsection{Baggrund}
Der er ikke stillet krav om dette, men (...)

\chapter{Brugertest}

Til at udføre vores brugertest skal vi bruge en eller flere personer
indenfor vores målgruppe. I kravspecifikationen er vores målgruppe
specificeret som:
\begin{quote}
En person i vores målgruppe er en hjemmesideskribent der er bekendt
med HTML. Personen har en professionel interessere i at teksten er
læsbar, således at vedkommende selv vil sætte sig ind i betydningen af
de udførte analysers resultater.
\end{quote}

I kravspecifikationen står der også at programmet skal være udformet
som et kommandolinjeværktøj og at det skal kunne køre på GNU--baserede
Linux maskiner. Dette stiller herved ét yderligere krav til
forsøgspersonerne: de skal have forudgående kendskab til
kommando\-linje\-baserede programmer.

\end{document}
