\documentclass[a4paper,oneside,article]{article}
\usepackage[T1]{fontenc}
\usepackage{textcomp}
\usepackage[utf8]{inputenc}
\usepackage[danish]{babel}
\usepackage[garamond]{mathdesign}


\renewcommand{\ttdefault}{pcr} % bedre typewriter font
\renewcommand{\rmdefault}{ugm} % garamond


\usepackage{lettrine}

%\overfullrule=5pt

\title{Konklusion på design--evaluering  \\
       \small{Førsteårsprojekt}}

\author
{
  Gruppe 1:\\
  Troels Henriksen (athas@sigkill.dk)\\
  Jesper Reenberg (reenberg@kampsax.dtu.dk)\\
  Martin Dybdal (dybber@dybber.dk)\\ \\
  Vejledere: Dina og Kasper
}


\pagestyle{plain}

\date{\today}

\begin{document}
\maketitle

Vi har selv prøvet at komme med noget efterkritik af vores
design--specifikation, da gruppen vi havde udvekslet med ikke havde
nogle konkrete kommentarer til vores design. Deres eneste kommentarer
var at vores dataflow--diagram måske skulle være et bilag og at der
manglede en forklaring af \textit{robots.txt}.

Vi havde selv fundet en enkelt ændring som vi vil lave. Til beregning
af både lixtal og FKRT skal bruges informationer om bl.a. antal ord og
antal sætninger. Det vil resultere i dobbeltarbejde hvis flere
analyser beregner disse tal, så vi vil lave en forudgående indsamling
af informationer om teksten, der kan være relevante for flere af
analyserne. Man kan også tænke sig at der skal tilføjes nye
tekstanalyser og det vil i den forbindelse også lette arbejdet med at
implementere disse.

\end{document}
