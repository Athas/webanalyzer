\documentclass[a4paper,oneside,article]{memoir}
\usepackage[T1]{fontenc}
\usepackage{textcomp}
\usepackage[utf8]{inputenc}
\usepackage[danish]{babel}
\usepackage[garamond]{mathdesign}
\usepackage{url}
\usepackage{graphicx}
\usepackage{pdflscape}
\usepackage{longtable}

\DeclareTextFontCommand{\textfleur}
{\fontencoding{T1}\fontfamily{FleurCornerCaps}\selectfont}


\renewcommand{\ttdefault}{pcr} % bedre typewriter font
\renewcommand{\rmdefault}{ugm} % garamond

\usepackage{lettrine}

%\overfullrule=5pt

%\setsecnumdepth{part}

\title{Hjemmesideanalyse  \\ \small{Førsteårsprojekt}}

\author
{
  Gruppe 1:\\
  Troels Henriksen (athas@sigkill.dk)\\
  Jesper Reenberg (reenberg@kampsax.dtu.dk)\\
  Martin Dybdal (dybber@dybber.dk)\\ \\
  Vejledere: Dina og Kasper
}


\setcounter{tocdepth}{2}
\setcounter{secnumdepth}{2}

\pagestyle{plain}

\date{\today}

\begin{document}
\maketitle
\newpage
\tableofcontents*
\newpage

\chapter{Indledning}

Mange hjemmesider forfattes i dag uden omtanke for om sidens målgruppe
er i stand til at læse sidens indhold --- det sproglige niveau er
ganske enkelt for højt. Ydermere forfattes indholdet af mange
hjemmesider med primitive værktøjer (f.eks. simple tekstfelter direkte
på siden, eller simple skriveprogrammer), der mangler de
hjælpefeatures som findes i almindelige skriveprogrammer. Dette
resulterer i at hjemmesider ofte indeholder flere stavefejl end
gennemsnitlig tekst. Disse mangler kan gøre hjemmesider sværere at
bruge for den tilsigtede målgruppe, og derved reducere deres
effektivitet, ligegyldigt hvad målet med siden så end er.

Baseret på denne problemstilling har vi implementeret et program
designet til at bistå hjemmesideforfattere med at forbedre den
sproglige kvalitet af deres sider, hvad enten dette er ved at reducere
antallet at stavefejl eller at omskrive unødigt kompliceret tekst.

Denne rapport beskriver design- og implementeringsprocessen bag vores
løsning af problemet, en arbejdsproces der blev udført i forbindelse
med Førsteårsprojektet på DIKU i blok 4, 2007.

Det forventes at læseren af denne rapport har en hvis grad af teknisk
kompetence, har erfaring med programmering og har en overordnet
forståelse for sproget HTML.

\chapter{Analyse}

\chapter{Programdesign}

\chapter{Implementation}

\chapter{Afprøvning}

\chapter{Konklusion}

\end{document}
